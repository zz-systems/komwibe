% !TeX document-id = {1220ecc9-4e94-4765-8bf8-191a286a997a}
\documentclass[
	ngerman,
	parskip=half,
	headsepline,
	fontsize=12pt,
	DIV=13,
	listof=leveldown,
	]{scrreprt}

\KOMAoptions{DIV=last}

\usepackage[utf8]{inputenc}
\usepackage[T1]{fontenc}

\usepackage{titling}

% Basic Look & Feel
\usepackage{babel}
\usepackage{lmodern}
\usepackage{microtype}
\usepackage{graphicx}
\usepackage{xspace}
\usepackage[xindy, nonumberlist, nogroupskip]{glossaries}
\usepackage{csquotes}
\usepackage{enumitem}
\usepackage{float}
\usepackage{gensymb, siunitx}
\sisetup{per-mode=symbol, binary-units=true}

% References
\usepackage[hyphens]{url}
\usepackage[hidelinks,linktoc=all]{hyperref}
\usepackage{cleveref}
\usepackage[backend=biber, style=trad-plain]{biblatex}
\addbibresource{bibliography.bib} 

% Tables
\usepackage{array}
\usepackage{booktabs}
\usepackage{multirow}

% Enable shell-escape
% !TeX TXS-program:compile = txs:///pdflatex/[--shell-escape]

%--------------------------------------------------------------------------------
%\lohead[]{}
%\rohead[]{\leftmark}
%\cohead[]{}

%
%captionsetup{format=plain,indention=2em}

%\makeglossaries
%\setacronymstyle{long-short}
%\loadglsentries{}

% Titel
\author{Jakob Birkenfeld, Josef Herbert, Sergej Zuyev}
\title{Datenbank ACM - Inhalt, Datenbankaufbau und Recherche}
\subject{Kommunikation in wissenschaftlicher und beruflicher Arbeit WS2017}
\titlehead{\includegraphics[width=\linewidth]{Logo_THM}}
\date{\today}

\graphicspath{{img/}}



%--------------------------------------------------------------------------------

\makeindex
\makeglossaries

\begin{document}
	\begin{titlepage}
		\maketitle
	\end{titlepage}
	
	\begin{abstract}
		In diesem Dokument wird die wissenschaftliche Datenbank \textbf{ACM} beschrieben und nach dem Datenbankfahrplan der UB Chemnitz \cite{resource:dbf} bewertet.
		
		In this document the scientific database \textbf{ACM} is described and evaluated according to \cite{resource:dbf}.
	\end{abstract}

	\clearpage
	
	\pagenumbering{Roman}
		\tableofcontents
		\listoffigures	
	\pagenumbering{arabic}
	
	\clearpage
	
	\chapter{Einführung}	
	Literaturdatenbanken ermöglichen die Recherche von Informationen im \textit{Deep Web} und somit den umfassenden Überblick über Publikationen einer Institution. Standardsuchmaschinen, wie z.B. \textit{Google}, erreichen diese Informationen oftmals nicht. Daher ist eine Nutzung von Literaturdatenbanken sehr sinnvoll. Die Digitalisierung hat dabei die früher üblichen Zettelkästen abgelöst und vereinfacht die Literaturrecherche enorm. Umfangreiche Suchfunktionen, teilweise sogar mit Volltextsuche, stehen dabei dem Benutzer zur Verfügung. Die in einer solchen Fachdatenbank erhaltenen Medientypen sind nicht auf Bücher beschränkt, sondern beinhalten oftmals auch Zeitschriften und multimediale Elemente. \cite{resource:wld}
	\ \\
	\ \\
	Die ACM (\textit{Association for Computing Machinery}) gilt mit ihrer Gründung im Jahr 1947 als erste wissenschaftliche Gesellschaft für Informatik \cite{resource:wacm}. Sie regt zum Dialog unter Forschern, Lehrenden und Fachleuten an \cite{resource:aacm}. Das geschieht durch:
\begin{itemize}
\item jährlich mehrere Fachkonferenzen.
\item \textit{sog. Special Interest Groups}, kurz \textit{SIG}. Diese Gruppen sind thematisch gegliedert und besitzen jeweils ein eigenes Leitungsgremium. Dadurch ist eine Spezialisierung der Mitglieder auf bestimmte Fachgebiete möglich.
\item regelmäßige Veröffentlichungen. Dazu zählen Magazine, Journals und Transactions.
\item eine digitale Bibliothek (\textit{Digital Libary}). Diese stellt Publikationen bis zum Gründungsjahr der ACM entgeltlich online zur Verfügung. Sie gilt als weltweit größte Sammlung ihrer Art.
\end{itemize}
Dieser Bericht erläutert und diskutiert die Möglichkeiten, die sich durch Nutzung der wissenschaftlichen Datenbank der ACM (\textit{Digital Libary}) eröffnen.
\ \\
\begin{figure}[ht]
\begin{minipage}[b]{0.55\linewidth}
\centering
\includegraphics[width=\textwidth]{img/acmLogo.PNG}
\caption{Das Logo der ACM.}
\end{minipage}
\end{figure}
% Bildquelle: \url{http://www.acm.org/images/top-menu/acm_logo_tablet.svg}

	\chapter{Methoden}
	Um einen möglichst umfassenden Überblick über die Datenbank der ACM zu erhalten, wird der Datenbankfahrplan der UB Chemnitz \cite{resource:dbf} verwendet. Dieser vereinfacht die Untersuchung der Möglichkeiten einer bibliographischen Datenbank anhand gezielter Fragestellungen.
\ \\	
\ \\
	Außerdem wird eine selbstständig und eine unselbstständig erschienene wissenschaftliche Publikation in der Datenbank recherchiert und untersucht. Der Schwerpunkt liegt dabei auf dem grundsätzlichen Aufbau und Inhalt der Datensätze.
	\chapter{Ergebnisse}
		
	\chapter{Diskussion}	
	
	\chapter{Eigenständigkeitserklärung}
	
	Wir versichern hiermit ehrenwörtlich, dass ich die vorliegende Arbeit selbstständig und ohne Benutzung anderer als der angegebenen Hilfsmittel angefertigt habe. Alle Stellen, die wörtlich oder sinngemäß aus veröffentlichten oder unveröffentlichten Quellen entnommen worden sind, sind als solche kenntlich gemacht. Diese Arbeit lag in gleicher oder ähnlicher Form noch keiner anderen Prüfungsbehörde vor und wurde bisher noch nicht veröffentlicht.
	
	Friedberg, den \today
	
	
	\rule[-0.2cm]{5cm}{0.5pt}
	
	\textsc{\theauthor} 

	
	
	
	\printbibliography
\end{document}
